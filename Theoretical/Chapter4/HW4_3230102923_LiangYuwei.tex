\documentclass[a4paper]{article}
\usepackage[affil-it]{authblk}

\usepackage[UTF8]{ctex}
\usepackage{amsmath}
\usepackage{amssymb}
\usepackage{geometry}
\usepackage{graphicx}

\geometry{margin=1.5cm, vmargin={0pt,1cm}}
\setlength{\topmargin}{-1cm}
\setlength{\paperheight}{29.7cm}
\setlength{\textheight}{25.3cm}

\newcommand{\esm}{\epsilon_M}
\newcommand{\fl}{\text{fl}}
\newcommand{\Eabs}{E_{\text{abs}}}
\newcommand{\Erel}{E_{\text{rel}}}
\newcommand{\abs}[1]{\left|#1\right|}
\newcommand{\normtwo}[1]{\left\|#1\right\|_2}
\begin{document}
% =================================================
\title{Numerical Analysis homework \# 3}

\author{梁育玮 Liang Yuwei 3230102923
  \thanks{Electronic address: \texttt{liangyuwei631@gmail.com}}}
\affil{(Mathematics and Applied Mathematics 2302), Zhejiang University }
\date{\today}

\maketitle


% ============================================
\section{Assignment}
\subsection*{I. Convert 477 to binary.}
we have $477 = 2^8 + 2^7 + 2^6 + 2^2 + 2^0 = (111011101)_2$. Thus in FPN, $477 = (1.11011101)_2 \times 2^8$.

\subsection*{II. Convert $\frac{5}{3}$ to binary.}
\begin{align*} 
  \frac{5}{3} &= (1.1010\dots)_2 \\
  &= (1.1010\dots )_2 \times 2^0
\end{align*}

\subsection*{III. Prove $x_R - x = \beta(x - x_L)$}
denote $\alpha = \beta -1$, which is the biggest digit in the FPN. Then we have
\[x_L = (\alpha.\alpha\alpha\ldots\alpha\alpha)_{\beta} \times \beta^{e-1},\quad x_R = (1.00\ldots 01)_{\beta} \times \beta^{e} .\]

Thus we have
\[x - x_L = \epsilon_M \beta^{e-1},\quad x_R - x = \epsilon_M \beta^{e}.\]

Therefore, $x_R - x = \beta(x - x_L)$.

\subsection*{IV. Two normalized FPNs adjacent to $x = \frac{5}{3}$}
The two normalized FPNs adjacent to $x = \frac{5}{3}$ are $x_L = (1.1010\dots101)_2 \times 2^0$ and $x_R = (1.1010\dots110)_2 \times 2^0$. 
Thus we have \begin{align*}
  x - x_L &= \epsilon_M (\frac{1}{4} + \frac{1}{16} + \frac{1}{64} + \dots)\\
&= \esm \frac{\frac{1}{4}}{1-\frac{1}{4}} = \frac{1}{3} \esm,\\
  x_R - x_L &= \esm\\
  \implies x_R - x &= \frac{2}{3} \esm.
\end{align*}  
Thus we have $x - x_L \le x_R -x$, which means $\fl(x) = x_L$.

Moreover, we have 
\[\Eabs(\fl(x)) = \frac{x- x_L}{x} = \frac{\esm}{5} = \frac{1}{5\times 2^{23}}\]

\subsection*{V. Unit roundoff}
Let $x$ be a float close to $1 + \esm$. As $x$ approaches $1 + \esm$ from the left, $fl(x) = 1$, and we have:
\[\Erel(\fl(x)) = \frac{x-1}{x} = \frac{\esm}{1 + \esm} \approx \esm\]
So the unit roundoff is $\esm$.

\subsection*{VI. Substraction $1 - \cos x$ when $x = \frac{1}{4}$}
We assume the opration is under the FPN system single precision.

$\cos \frac{1}{4} = (1.111 1000 0000 1010 1010 0101)_2 \times 2^{-1}$
\begin{align*}
  1 - \cos \frac{1}{4} &= (1.000 0000 0000 0000 0000 0000)_2 \times 2^0 - (1.111 1000 0000 1010 1010 0101)_2 \times 2^{-1}\\
&= (1.000 0000 0000 0000 0000 0000)_2 \times 2^0 - (0.1111 1000 0000 1010 1010 0101)_2 \times 2^{0}\\
&=(1.110 0000 1010 1011 1111 1110) \times 2^{-7}.
\end{align*}

so 7 bits of precision are lost.

\subsection*{VII. Avoid catastrophic cancellation}
there are two ways to avoid catastrophic cancellation:
\subsubsection*{Taylor series}
\[  1 - \cos x = \frac{x^2}{2} - \frac{x^4}{4!} + \frac{x^6}{6!} + \dots\]

\subsubsection*{Use trigonometric identity}
\begin{align*}
  1- \cos x &=  2\sin^2\frac{x}{2}\\
&= 2\left(\frac{x}{2} - \frac{x^3}{48} + \frac{x^5}{3840} + \dots\right)^2
\end{align*}

\subsection*{VIII. condition number}
\subsubsection*{(1) $f = (x-1)^\alpha$}
The condition number is
\[
  C_f(x) = \abs{\frac{\alpha x}{x-1}}
\]
And when $x$ is close to 1, $C_f(x)$ is large.

\subsubsection*{(2) $f = \ln x$}
The condition number is
\[
  C_f(x) = \frac{1}{\abs{\ln x}}
\]
And when $x$ is close to 1, $C_f(x)$ is large.

\subsubsection*{(3) $f = e^x$}
The condition number is
\[
  C_f(x) = \abs{x}
\]
And when $x$ is large, $C_f(x)$ is large.

\subsubsection*{(4) $f = \arccos {x}$}
The condition number is
\[
  C_f(x) = \abs{\frac{x}{\arccos {x}\sqrt{1-x^2}}}
\]
And when $x$ is close to 1 or -1, $C_f(x)$ is large.

\subsection*{IX. Consider the function $f(x) = 1-e^{-x}$ for $x\in [0,1]$}
\subsubsection*{IX-a}
The condition number is\[ 
\text{cond}_f(x) = \abs{\frac{xe^{-x}}{1-e^{-x}}} = \frac{x}{e^{x} - 1}.
\]

Then, we need to show $x \leq e^x - 1$ for $x \in [0,1]$, which is trival. Thus we have $\text{cond}_f(x) \leq 1$.

\subsubsection*{IX-b}
Assume there exists $x_A$ such that \[
1-e^{-x_A} = (1-e^{-x})(1+\delta).  \]

Then we have \[
\delta = \frac{e^{-x} - e^{-x_A}}{1-e^{-x}} = \frac{1 - e^{x-x_A}}{e^{x} - 1} \approx \frac{x_A - x}{e^x - 1}.\]

Therefore, The condition number for algorithm $A$ is \begin{align*}
  \text{cond}_A(x) &= \frac{1}{\epsilon_u}\abs{\frac{x_A-x}{x}}\\
&= \frac{1}{\epsilon_u}\frac{\delta(e^x-1)}{x} \\&\leq \frac{e^x-1}{x}
\end{align*}

\subsubsection*{IX-c}
The following is the figures of $\text{cond}_f(x)$ and $\text{cond}_A(x)$(Figure \ref{fig:1}). We can see that these when $x$ is close to 0, the condition numbers are close. But when $x$ is close to 1, the condition number of $f(x)$ is much smaller than that of $A$, since the condition number of $f(x)$ is decreasing while the condition number of $A$ is increasing. 
\begin{figure}[ht]
  \centering
  \includegraphics[width=0.6\textwidth]{1.png}
  \caption{The condition number of $f(x)$ and $A(x)$} 
  \label{fig:1}
\end{figure}

\subsection*{X. Prove lemma 4.68}
Assume the singular value of $A$ is \[0 \leq \sigma_1 \leq \sigma_2 \leq \dots \leq \sigma_n\]. Then we have 
\[\sigma_1^2 x^T x \leq x^TA^TAx\leq \sigma_n^2 x^Tx,\]
where the equality holds when $x$ is the eigenvector of $A^TA$ corresponding to $\sigma_1^2$ and $\sigma_n^2$.

So we have
\[
 \normtwo{A} = \max_{\normtwo{x} = 1} \normtwo{Ax} = \max_{\normtwo{x} = 1} \sqrt{x^TA^TAx} = \sigma_n,
 \]

And the maxium singular value of $A^{-1}$ is $\sigma_1^{-1}$, so we have
\[
  \normtwo{A^{-1}} = \sigma_1^{-1}.
\]

Therefore, we have $\text{cond}_2(A) = \frac{\sigma_n}{\sigma_1} = \frac{\sigma_{\max}}{\sigma_{\min}}$. When $A$ is normal, the singular values of $A$ is its eigenvalues, so we have $\text{cond}_2(A) = \frac{\lambda_{\max}}{\lambda_{\min}}$. And when $A$ is unitary, the singular values of $A$ are all 1, so we have $\text{cond}_2(A) = 1$.
\subsection*{XI. Wilkinson's polynomial}
We can compute the component of componentwise condition number of $f$ is\[
b_{j+1}(x) = \abs{\frac{a_j\frac{r^j}{\sum_{i =1}^{n}ia_ix^i-1}}{r}}=\abs{\frac{a_jr^j}{\sum_{i =1}^{n}ia_ir^i}}.
\]  
Thus The componentwise condition number of $f$ is \[
\text{cond}_{f}(x) = \sum_{j=1}^{n}b_{j}(x) = \dfrac{\sum_{i = 0}^{n-1}\abs{a_i r^i}}{\abs{\sum_{i =1}^{n}ia_ir^i}}.
\]

When $f(x) = \prod_{k=1}^{p}(x-k)$, 

\subsection*{XIII. bisection method}
A single precision floating point number has 24 bits of precision, and to obtain absolute accuracy of $10^{-6}$, we need 20 significant digits for fraction, and only 4 left for integer part. Thus only when the absolute value of root is less than 16 can we obtain the absolute accuracy of $10^{-6}$.

when the root is greater than 16, eg 128. We can not conpute the root with absolute accuracy less than $10^{-6}$.

\subsection*{XIV. Fitting curves}
\end{document}
