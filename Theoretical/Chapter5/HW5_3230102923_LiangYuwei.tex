\documentclass[a4paper]{article}
\usepackage[affil-it]{authblk}

\usepackage[UTF8]{ctex}
\usepackage{amsmath}
\usepackage{amssymb}
\usepackage{geometry}
\usepackage{graphicx}

\geometry{margin=1.5cm, vmargin={0pt,1cm}}
\setlength{\topmargin}{-1cm}
\setlength{\paperheight}{29.7cm}
\setlength{\textheight}{25.3cm}

\newcommand{\esm}{\epsilon_M}
\newcommand{\fl}{\text{fl}}
\newcommand{\Eabs}{E_{\text{abs}}}
\newcommand{\Erel}{E_{\text{rel}}}
\newcommand{\abs}[1]{\left|#1\right|}
\newcommand{\normtwo}[1]{\left\|#1\right\|_2}
\newcommand{\norminf}[1]{\left\|#1\right\|_{\infty}}
\newcommand{\norm}[1]{\left\|#1\right\|}
\newcommand{\innerProduct}[2]{\langle #1 , #2\rangle}

\begin{document}
% =================================================
\title{Numerical Analysis homework \# 5}

\author{梁育玮 Liang Yuwei 3230102923
  \thanks{Electronic address: \texttt{liangyuwei631@gmail.com}}}
\affil{(Mathematics and Applied Mathematics 2302), Zhejiang University }
\date{\today}

\maketitle


% ============================================
\subsection*{I. proof of Theorem 5.7}
To show that $\mathcal{c}[a,b]$ is a inner-proudct space over \(\mathbf{C}\) with given inner product, we need to prove that 
the given inner product satisfy the three principle
\begin{itemize}
    \item \textbf{Linearity}: \begin{align*}
        \langle \alpha u_1 + \beta u_2,v \rangle& = \int_a^b \rho(t) \left(\alpha u_1(t) + \beta u_2(t) \right) \overline{v(t)}dt\\   
        & = \alpha\int_{a}^{b} \rho(t) u_1(t)\overline{v(t)} dt + \beta \int_{a}^{b} \rho(t) u_2(t)\overline{v(t)} dt
       \\ & = \alpha  \langle u_1, v  \rangle + \beta \langle u_2 , v \rangle
    \end{align*}\item \textbf{Positivity}: \[
        \innerProduct{u}{u} = \int_{a}^{b} \rho(t) u(t) \overline{u(t)} dt= \int_{a}^{b} \rho(t) \norm{u(t)}^2 dt
                     \geq 0\]
Equality holds if and only if \(u(t) \equiv 0\) in \([a,b]\) since \(u(t)\) is continous.
    \item \textbf{Hermitian symmetry}:        \[ \innerProduct{v}{u} =\int_{a}^{b} \rho(t) v(t) \overline{u(t)} dt 
    = \overline{\int_{a}^{b} \rho(t) u(t) \overline{v(t)} dt} = \overline{\innerProduct{u}{v}} \]
\end{itemize}
Moreover, \(\mathcal{C}[a,b]\) is a normed vector space under the norm induced by the inner product.

\subsection*{II. Consider the Chebyshev polynomial of the first kind}
\subsubsection*{II-a}
when \( m \neq n \), we have 
\begin{align*}
    \int_{-1}^{1}\frac{\cos(n\arccos{x})\cos(m\arccos{x})}{\sqrt{1-x^2}}dx  &= -\int_{-1}^{1} \cos(n\arccos{x})\cos(m\arccos{x}) 
    d(\arccos x) \\   & = \int_{0}^{\pi} \cos (mt) \cos(nt) dt \\& = 0
\end{align*}
the last eqaution is hold by the Fourier Series Theory.
\subsection*{II-b}
We have \(T_0(x) = 1\), \(T_1(x) = x\) and \(T_2(x) = 2x^2-1\). And we can compute that 
\[\innerProduct{T_0}{T_0} =\int_{0}^{\pi} 1^2 dt = \pi  \]
\[\innerProduct{T_1}{T_1} = \int_{0}^{\pi} \cos^2t =\frac{\pi}{2} \]
\[\innerProduct{T_2}{T_2} = \int_{0}^{\pi} \cos^2(2t)dt = \frac{\pi}{2}\]
so we can noralize them as \({S_0(x)} = \frac{1}{\sqrt{\pi}}\), \(S_1(x) = \sqrt{\frac{2}{\pi}}x\) and 
\({S_2(x)} = \sqrt{\frac{2}{\pi}}(2x^2-1)\)

\subsection*{III. Least-Square approximation of a continous function}
\subsubsection*{III-a. Fourier Expansion}
Choose a orthonormal base for quadratic polynomial under the weight function first, Let \({S_0(x)} = \frac{1}{\sqrt{\pi}}\), \(S_1(x) = \sqrt{\frac{2}{\pi}}x\) and 
\({S_2(x)} = \sqrt{\frac{2}{\pi}}(2x^2-1)\)

We have \begin{align*}
    &\innerProduct{y}{S_0} =\int_{-1}^{1}\frac{1}{\sqrt{\pi}}dx = \frac{2}{\sqrt{\pi}},\\
    &\innerProduct{y}{S_1} =\int_{-1}^{1}\sqrt{\frac{2}{\pi}}xdx = 0\\
    &\innerProduct{y}{S_2} =\int_{-1}^{1}\sqrt{\frac{2}{\pi}}(2x^2-1)dx = -\frac{2}{3}\sqrt{\frac{2}{\pi}}
\end{align*}
so the approximation polynomial is \(\hat{y}_1(x) =\innerProduct{y}{S_0}S_0 + \innerProduct{y}{S_1}S_1 + \innerProduct{y}{S_2}S_2 = -\frac{8}{3\pi}
x^2 + \frac{10}{3\pi}  \)

\subsubsection*{III-b. normal eqaution}
We first construct the Gram matrix
\[
G(1,x ,x^2) =
\begin{bmatrix}
    \innerProduct{1}{1} &\innerProduct{1}{x}&\innerProduct{1}{x^2}\\
    \innerProduct{x}{1} &\innerProduct{x}{x}&\innerProduct{x}{x^2}\\
    \innerProduct{x^2}{1} &\innerProduct{x^2}{x}&\innerProduct{x^2}{x^2}\\
\end{bmatrix}=
\begin{bmatrix}
    \pi     &0      &\frac{\pi}{2}     \\
    0       &\frac{\pi}{2}       &0    \\
    \frac{\pi}{2}        &0      &\frac{3\pi}{8}     \\
\end{bmatrix}.
\]
We then calculate the vector
\[
\mathbf{c} = \begin{bmatrix}
    \innerProduct{y}{1} \\ \innerProduct{y}{x}\\ \innerProduct{y}{x^2}
\end{bmatrix} = \begin{bmatrix}
    2\\0\\\frac{2}{3}
\end{bmatrix}
\]
then we have\[
\begin{bmatrix}
    a_0\\a_1\\a_2
\end{bmatrix} = G^{-1}\mathbf{c} = \begin{bmatrix}
    \frac{10}{3\pi} \\ 0 \\ -\frac{8}{3\pi}
\end{bmatrix}
\]
so \(\hat{y}_2(x) = -\frac{8}{3\pi}
x^2 + \frac{10}{3\pi},\) which is same with the result in (a).

\subsection*{IV. Discrete Least Square Problem}
\subsubsection*{IV-a}
set \begin{align*}
&u_1 = 1\\
&u_2 = x - \frac{\innerProduct{u_1}{x}}{\innerProduct{u_1}{u_1}}\times 1 = x - \frac{13}{2}\\
&u_3 = x^2 - \frac{\innerProduct{u_1}{x^2}}{\innerProduct{u_1}{u_1}}\times 1 - \frac{\innerProduct{u_2}{x^2}}{\innerProduct{u_2}{u_2}}\times u_2 = x^2 - 13(x-\frac{13}{2}) - \frac{325}{6}
\end{align*}
noralize them as \begin{align*}
    &v_1 = 0.28868\\\
&v_2 = 0.083624x - 0.54356\\
&v_3 =0.027372 x^2 - 0.35584 x + 0.83030
\end{align*}
\subsubsection*{IV-b}
set \(y\) as sales record
\begin{align*}
    &\innerProduct{y}{v_1} = 479.78 \\\
&\innerProduct{y}{v_2} = 49.255\\
&\innerProduct{y}{v_3} =330.33
\end{align*}
so \(\hat{\varphi}(x) = 479.78 v_1 + 49.255 v_2 + 330.33 v_3\), therefore, \(\hat{\varphi}(x) = 9.04x^2 -113.43x + 441.23\),which is 
same with the result in Example 5.55.

\subsubsection*{VI-c}
The part (a) can be reused, we can use the same orthonormal polynomial, but the part (b) can not be reused, we must calculate parameter once omre.

When cope with a large number of records table, we can compute orthonormal polynomial only once and use it permanently, which is more efficient than normal equation.   
\end{document}