\documentclass[a4paper]{article}
\usepackage[affil-it]{authblk}

\usepackage[UTF8]{ctex}
\usepackage{amsmath}
\usepackage{amssymb}
\usepackage{geometry}
\usepackage{graphicx}

\geometry{margin=1.5cm, vmargin={0pt,1cm}}
\setlength{\topmargin}{-1cm}
\setlength{\paperheight}{29.7cm}
\setlength{\textheight}{25.3cm}


\begin{document}
% =================================================
\title{Numerical Analysis homework \# 6}

\author{梁育玮 Liang Yuwei 3230102923
  \thanks{Electronic address: \texttt{liangyuwei631@gmail.com}}}
\affil{(Mathematics and Applied Mathematics 2302), Zhejiang University }
\date{\today}

\maketitle
\section*{I. Simpson's rule}
\subsection*{I-a}
We have the divided differneces table
\[
\begin{array}{c|cccc}
    -1 & y(-1)&&&\\
    0&  y(0)&   y(0)-y(-1)&    &   \\
    0&  y(0)&   y'(0)&  y'(0)-y(0) + y(-1)   &\\
    1&  y(1)&   y(1)-y(0)   &y(1)-y(0)-y'(0)&   \frac{y(1) - y(-1) - 2y'(0)}{2}
\end{array} 
\]
\begin{align*}
    p_3(y;-1,0,0,1;t) &= y(-1) + [y(0)-y(-1)](t+1) + [y'(0) - y(0) + y(-1)]t(t+1) + \frac{y(1) - y(-1) - 2y'(0)}{2}t^2(t+1) \\
& = y(0) + [y'(0)+y(-1)]t + [\frac{y(-1)+y(1)}{2}-y(0)]t^2 + \frac{y(1) - y(-1) - 2y'(0)}{2}t^3
\end{align*} 
so \[\int_{-1}^{1}p_3(y;-1,0,0,1;t)dt = y(0) + [\frac{y(-1)+y(1)}{2}-y(0)]\frac{1}{3} = \frac{1}{6}[y(-1) + 4y(0) + y(1)]\]
which is equivalent to Simpson's rule.
\subsection*{I-b}
we have \[R(y,t) = y(t) - p(y;-1,0,0,1;t) = \frac{y^{(4)}(\xi(t))}{4!}t^2(t+1)(t-1)\]
\begin{align*}
    E^s(y) &= \int_{-1}^{1} \frac{y^{(4)}(\xi(t))}{4!}t^2(t+1)(t-1) dt \\
    & = \frac{y^{(4)}(\zeta)}{4!}\int_{-1}^{1}t^2(t+1)(t-1) dt \\
    & = -\frac{y^{(4)}(\zeta)}{90}
\end{align*}

\subsection*{I-c}
for \(f \in \mathcal{C}^4[a,b]\), Let \(g(t) = f(\frac{2t}{b-a} + \frac{a+b}{2})\), then \(g \in \mathcal{C}^4[-1,1]\), thus 
\begin{align*}
    \int_{a}^{b}f(t)dt &= (b-a)\int_{-1}^{1}g(s)ds \\
    &= \frac{b-a}{6}(g(-1) + 4g(0) + g(1)) - \frac{(b-a)}{90}g^{(4)}(\xi)\\
& = \frac{b-a}{6}(f(a) + 4f(\frac{a+b}{2}) + f(b)) - \frac{(b-a)^5}{180}f^{(4)}(\zeta)
\end{align*} 
Set  \(h  =\frac{b-a}{n}\), \(x_k = a + kh,\quad k = 0,1,\ldots,n\)

We can get\[\int_{x_{2k}}^{x_{2k+2}}f(t)dt = \frac{2h}{6}(f(x_{2k})+ 4f(x_{2k+1}) + f(x_{2k+2})) - \frac{h^5}{90}f^{(4)}(\zeta_k),\quad k = 0,1,\ldots,\frac{n}{2}-1.\]
Sum them together, we have 
\begin{align*}
    \int_{a}^{b}f(t)dt = &\frac{b-a}{6}(f(x_0) + 4f(x_1) + 2f(x_2) + 4f(x_3) + 2f(x_4) + \ldots + 4f(x_{n-1}) + f(x_n))    \\
    &-\frac{b-a}{180}h^4\left( \frac{f^{(4)}(\zeta_1) + f^{(4)}(\zeta_2) + \ldots + f^{(4)}(\zeta_{n/2})}{n/2} \right) 
\end{align*}
so we get the composite Simpson's tule, and the reminder is
\[E^S_n(f) =-\frac{b-a}{180}h^4\left( \frac{f^{(4)}(\zeta_1) + f^{(4)}(\zeta_2) + \ldots + f^{(4)}(\zeta_{n/2})}{n/2} \right)  
= -\frac{b-a}{180}h^4f^{(4)}(\zeta)\]
the last eqaution holds due to the fact \(f^{(4)}(x)\) is continous on \([a,b]\)

\section*{II. The number of subinterval requeired to approximation}
\subsection*{II-a. composite trapezodial rule}
Set \(f(x) = e^{-x^2}\), then \(f''(x) = 4x^2e^{-x^2} \leq \frac{4}{e}\), the absolute error is 
\[ \left|E_n^T(f)\right|  = \left|\frac{1}{12}\frac{1}{n^2}f''(\xi)\right| \leq \frac{1}{3en^2}\leq0.5\times 10^6\]
then we have \(n \geq 300.37\), so \(n = 301\)
\subsection*{II-b. composite Simpson's rule}
we have \(f^{(4)} (x) = 16 x^4 e^{-x^2} \leq \frac{16}{e} \quad \forall x \in [0,1]\), the absolute error is
\[|E^S_n(f) |= \left|\frac{1}{180}\frac{1}{n^4}f^{(4)}(x)\right| \leq \left|\frac{1}{180}\frac{1}{n^4}\frac{16}{e}\right|\leq0.5\times 10^6 \]
we have \(n\ge 15.99\), so \(n = 16\)

\section*{Gauss-Laguerre quadrature formula}
\section*{III-a}
Let \(p(t) = 1\) and \( p(t) = t\) respetively, then we have 
\[\int_{0}^{+\infty}(t^2 + at + b)e^{-t} dt= 0,\qquad \int_{0}^{+\infty} (t^3 + at^2 + bt)e^{-t} = 0.\]
so we have \[
\begin{cases}
    2+a+b = 0\\
    6+2a+b=0
\end{cases}\implies\begin{cases}
    a = -4\\b=2
\end{cases}
\]
so \(\pi_2(t) = t^2 -4t+2\)

\subsection*{III-b}
The zeros of \(\pi_2\) is \(t_1 = 2+\sqrt{2}\) and \(t_2  = 2-\sqrt{2}\), let \(f(t) = 1\) and \(f(t) = t\) respetively, then we have
\[\begin{cases}
    1 = w_1 + w_2\\ 1  = (2+\sqrt{2}) w_1 + (2-sqrt) w_2
\end{cases}
\implies\begin{cases}
    w_1 = \frac{2-\sqrt{2}}{4}\\w_2=\frac{2+\sqrt{2}}{4}
\end{cases}
\]
so we have \[\int_{0}^{+\infty}f(t)dt = \frac{2-\sqrt{2}}{4}f(2+\sqrt{2}) + \frac{2+\sqrt{2}}{4} f(2-\sqrt{2})+ E_2(f)\]

by theorem 6.33, we have \[E_2(f) = \frac{f^{(4)}(\tau)}{4!}\int_{0}^{+\infty}(t^2 -4t+2)^2e^{-t}dt = \frac{f^{(4)}(\tau)}{6}\]

\subsection*{III-c}
we have \begin{align*}
I_2(f) = &\frac{2-\sqrt{2}}{4}f(2+\sqrt{2}) + \frac{2+\sqrt{2}}{4} f(2-\sqrt{2}) \\
& =\frac{2-\sqrt{2}}{4}\frac{1}{3+\sqrt{2}} + \frac{2+\sqrt{2}}{4} \frac{1}{3-\sqrt{2}}\\
& = \frac{4}{5}
\end{align*}
Thus \[E_2(f) = 0.8 - 0596347361 = 0.20365263 = \frac{\frac{4!}{(1+\tau)^5}}{6} = \frac{4}{(1+\tau)^5}\]
So, \(\tau = 0.8139862\)

\section*{IV. reminder of Gauss formula}
\subsection*{IV-a}
we have \(h(m) = 1, h'(m) = 0, q(m) = 0, q'(m) = 1\), and 
\begin{align*}
h'(m) = b_m \ell^2_m(t) + 2(a_m  + b_m t)\ell_m(t)\ell'_m(t) \\
\end{align*}
thus\[
\begin{cases}
    a_m + b_m x_m = 1\\
    b_m + 2(a_m + b_m x_m)\ell'_m(x_m) = 0
\end{cases}\implies
\begin{cases}
    a_m = 1  + \frac{x_m}{2\ell'_m(x_m)}\\
    b_m = -\frac{1}{2\ell'_m(x_m)}
\end{cases}
\]
where \(\ell'_m(x_m) = \sum_{i \neq m} \frac{1}{x_m-x_i}w\).

Similarly, we have
\[
\begin{cases}
    c_m + d_m x_m = 0\\
    d_m + 2(c_m + d_m x_m)\ell'_m(x_m) = 1
\end{cases}
\implies
\begin{cases}
    c_m = -x_m\\
    d_m = 1
\end{cases}
\]

\subsection*{IV-b}
Set \[
w_k = \int_{a}^{b} h_k(t) \rho(t) dt,\quad \mu_k = \int_{a}^{b} q_k(t) \rho(t)dt.
\]
Then for all \(p \in \mathbb{P}_{2n-1}\)
\begin{align*}
    \int_{a}^{b} p(t)
    \rho(t) dt& = \int_{a}^{b} \sum_{k=1}^{n} [h_k(t)f_k + q_k(t)f'_k] \rho(t) dt\\
    & = \sum_{k = 1}^{n} [f_k\int_{a}^{b} h_k(t) \rho(t)dt +f'_k\int_{a}^{b} h_k(t) \rho(t)dt ]\\
    & = \sum_{k = 1}^{n} [w_k \rho(t)dt +v_k \mu_k(t)] \rho(t)dt 
\end{align*}

\subsection*{IV-c}
To make \(\mu_k = 0\) for all \(k\), we have
\begin{align*}
    0 =& \int_{a}^{b} q_k(t) \rho(t)dt\\ 
 \implies   0=&\int_{a}^{b}\rho(t)(t-x_k)\prod_{i\neq k}(t-x_i)^2 dt\\
    =&\int_{a}^{b}\rho(t)\prod_{i = 1}^{n} (t-x_i) \prod_{i\neq k}(t-x_i) dt
\end{align*}
Since \(\prod_{i\neq k}(t-x_i), k = 1,2\ldots,n\) form a basis of \(\mathbb{P}_{n-1}\), \(\mu_k = 0\) for each 
\(k = 1,2,\ldots, n\) is equivalent to
\[
\int_{a}^{b}p(t)\rho(t)\prod_{i = 1}^{n}(t-x_i)dt = 0,\quad \forall p \in \mathbb{P}_{n-1}
\]  




% ===================================================
\end{document}