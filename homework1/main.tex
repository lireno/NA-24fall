\documentclass[a4paper]{article}
\usepackage[affil-it]{authblk}
\usepackage[backend=bibtex,style=numeric]{biblatex}

\usepackage{amsmath}
\usepackage[UTF8]{ctex}
\usepackage{geometry}
\geometry{margin=1.5cm, vmargin={0pt,1cm}}
\setlength{\topmargin}{-1cm}
\setlength{\paperheight}{29.7cm}
\setlength{\textheight}{25.3cm}

\addbibresource{citation.bib}

\begin{document}
% =================================================
\title{Numerical Analysis homework \# 1}

\author{梁育玮 3230102923
  \thanks{Electronic address: \texttt{liangyuwei631@gmail.com}}}
\affil{(数应2302), Zhejiang University }


\date{Due time: \today}

\maketitle



% ============================================
\section*{I. The interval in bisection method}

\subsection*{I-a}

设第n步迭代之后的区间长度为$h_n$, $h_0 = 2$\\
\hspace*{2em}则$h_{n+1} = \frac{1}{2}h_n$,所以$h_n = \dfrac{1}{2^n}h_0 = \dfrac{1}{2^n}2 = \dfrac{1}{2^{n-1}}$
\\\hspace*{2em}即第n步迭代之后的区间长度为$\dfrac{1}{2^{n-1}}$。
\subsection*{I-b}
设第n步迭代之后区间 $[a_n, b_n]$ 的中点为$c_n$,则$c_n = \dfrac{a_n + b_n}{2}$。\\
\hspace*{2em}则$ |r - c_n| \leq \dfrac{b_n - a_n}{2} = \dfrac{h_n}{2} = \dfrac{1}{2^{n}}$

\section*{II. \emph{reltative error} in bisection method}
当迭代到第n步之后,区间长度$h_{n+1} = \dfrac{b_0 - a_0}{2^n}$,
\[相对误差E_{rel} = \dfrac{|r - c_n|}{|r|} \leq \dfrac{\dfrac{a_0 - b_0}{2^{n+1}}}{a_0} =  \dfrac{a_0 - b_0}{2^{n+1} a_0}
\]
\hspace*{2em}当$
n \geq \dfrac{\log(b_0 - a_0) - \log \epsilon - \log a_0}{\log 2} - 1.
$时,有$E_{rel} \leq \epsilon$。

\section*{III. Newton's method for polynoial} 
\[
p(x) = 4x^3 - 2x^2 + 3 = 0
\]
求导得
\[
p'(x) = 12x^2 - 4x
\]
\[
\begin{array}{|c|c|c|c|}
\hline
n & x_n & p(x_n) & p'(x_n) \\
\hline
0 & -1 & -3 & 16 \\
1 & -0.8125 & -2.6816 & 9.8438 \\
2 & -0.5401 & -0.7727 & 6.5685 \\
3 & -0.4224 & -0.1319 & 5.1706 \\
4 & -0.3969 & & \\
\hline
\end{array}
\]

\section*{IV. Newton's method in which only the derivative at $x_0$ is used}

\begin{align*}
    x_{n+1} &= x_n - \frac{f(x_n)}{f'(x_0)} \\  
    x_{n+1} - \alpha &= x_n - \alpha - \frac{f(x_n)}{f'(x_0)} \\ 
    x_{n+1} - \alpha &= (x_n - \alpha) \left( 1 - \frac{f(x_n) - f( \alpha)}{f'(x_0)(x-\alpha)} \right)\\
    e_{n+1} &= e_n \left(1 - \frac{f'(\xi _n)}{f'(x_0)} \right)
\end{align*}
其中$\xi _n$与$x_n$有关, 令 $s = 1, C = \left(1 - \frac{f'(\xi _n)}{f'(x_0)} \right)$即可

\section*{V. the iteration $x_{n+1} = \tan ^{-1} x_n$}
(i)若$x_1 = 0$,结论是显然的\\
(ii)若$x_1 \neq 0$,由归纳易知 $x_n \neq 0 $ \\
令 $$f(x) = \tan ^{-1} x $$ 则$$|f'(x)| = \left| \frac{1}{1+x^2}\right| < 1$$
由压缩映射定理,数列$\{x_n\}$收敛

\section*{VI. continued fraction}
令$\alpha$为方程$x^2 + px -1 = 0$大于0的解,则有$\alpha^2 + p\alpha -1 = 0$\\
\[\alpha =  \frac{-p + \sqrt{p^2+4}}{2}
\]
注意到
\[
x_{n+2} = \frac{1}{p + \dfrac{1}{p + x_n}}= \frac{p + x_n}{p^2 + px_n + 1}
\]
\begin{align*}
  x_{n+2} - \alpha &= \frac{p + x_n}{\left(p^2 + px_n + 1\right)} - \alpha \\
  &= \frac{p + x_n - p^2\alpha - px_n\alpha - \alpha}{\left(p^2 + px_n + 1\right)}\\
  &= \frac{p(1 - p\alpha) + x_n- px_n\alpha - \alpha}{\left(p^2 + px_n + 1\right)}\\
  &= \frac{p\alpha^2 + x_n- px_n\alpha - \alpha}{\left(p^2 + px_n + 1\right)}\\
  &= \frac{(x_n - \alpha)(1-p\alpha)}{\left(p^2 + px_n + 1\right)}  
\end{align*}
其中$1 > 1 - p\alpha = \alpha^2 > 0$,故$|x_{n+2} - \alpha| < |x_n - \alpha|$\\
从而数列$\{x_1, x_3, x_5, \dots \}$单调递减且有下界,故存在极限,设极限为x(x>0),则有
\[
  x = \frac{p + x}{p^2 + px + 1}
\]
化简得
\[
  x^2 + px -1 = 0
\]
从而\[x = \alpha\]
同理可证数列$\{x_2, x_4, x_6, \dots \}$极限也为$\alpha$\\
综上可知数列$\{x_n\}$的极限为$\alpha$

\section*{VII. $a_0 < 0 < b_0$}
此时相对误差不一定能得到一个很好的测量,若根$r$很接近0的话,相对误差会很大。极端情况$r = 0$,相对误差不能得到一个很好的估计 

\section*{VIII. Newton's method at a root of multipicity k}
\subsection*{VIII-a}
若零点 $\alpha$ 是多重根,则点列$\{x_n\}$将线性收敛到根 $\alpha$,而不是二阶收敛,
我们可以通过计算$\dfrac{|x_{n+1} - \alpha|}{|x_n - \alpha|}$来判断,若其趋近于0,则为二阶收敛;反之为一阶收敛,$\alpha$为多重根

\subsection*{VIII-b}
设$f(x) = (x - \alpha)^k g(x)$\\
其中\[g(x)=
\begin{cases}
  \frac{f(x)}{(x-\alpha)^k} &\text{if } x \neq \alpha\\
  \frac{f^{(k)}(\alpha)}{k!}&\text{if } x = \alpha
\end{cases}
\] 

易知$g(x)$连续,且\[
f'(x) = k(x-\alpha)^{k-1}g(x) + (x-\alpha)^kg'(x)
\]
\begin{align*}
  x_{n+1} &= x_n -k\frac{f(x)}{f'(x)}\\
  \implies x_{n+1} - \alpha&= x_n - \alpha -k\frac{(x - \alpha)^k g(x)}{k(x-\alpha)^{k-1}g(x) + (x-\alpha)^kg'(x)}\\
  \implies \frac{x_{n+1} - \alpha}{x_n - \alpha} &= 1 - \frac{kg(x)}{kg(x)+ (x-a)g'(x)}\\
  \implies \frac{x_{n+1} - \alpha}{(x_n - \alpha)^2} &= \frac{g'(x)}{kg(x)+ (x-a)g'(x)}\\
  \implies \left|\frac{x_{n+1} - \alpha}{(x_n - \alpha)^2} \right| &\leq \frac{g'(x)}{kg(x)}
\end{align*}
由于$g(x)$连续,故$\lim\limits_{x \to \alpha} g(x) = g(\alpha)$,故存在$\delta_1 > 0$使得当$|x - \alpha| < \delta$时,有$|g(x) - g(\alpha)| < \dfrac{g(\alpha)}{2}$\\
因为$f(x) \in C^{k+1},f(x)$在$\alpha$处的展开式为
\[
f(x) = \frac{f^{(k)}(\alpha)}{k!}(x-\alpha)^k + \frac{f^{(k+1)}(\xi)}{(k+1)!}(x-\alpha)^{k+1}
\]
其中$\xi$在$x$和$\alpha$之间
\begin{align*}
    g'(\alpha) &= \lim\limits_{x \to \alpha} \frac{g(x) - g(\alpha)}{x - \alpha}\\
    &= \lim\limits_{x \to \alpha} \dfrac{\frac{f(x)}{(x-\alpha)^k}-\frac{f^{(k)}(\alpha)}{k!}}{x - \alpha}\\
    &= f^{(k+1)}(\xi)
  \end{align*}
从而g'(x)在$\alpha$处连续,故存在$\delta_2 > 0$使得当$|x - \alpha| < \delta$时,有$|g'(x) - g'(\alpha)| < \dfrac{g'(\alpha)}{2}$\\
,故有
\[
\left|\frac{x_{n+1} - \alpha}{(x_n - \alpha)^2} \right| \leq \frac{3g'(\alpha)}{kg(\alpha)}
\]
从而$\{x_n\}$二阶收敛
% ===============================================


\end{document}