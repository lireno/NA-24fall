\documentclass[a4paper]{article}
\usepackage[affil-it]{authblk}
\usepackage[backend=bibtex,style=numeric]{biblatex}
\usepackage{graphicx}
\usepackage{amsmath}
\usepackage{geometry}
\usepackage{array}
\usepackage{tcolorbox}
\usepackage{booktabs}
\geometry{margin=1.5cm, vmargin={0pt,1cm}}
\setlength{\topmargin}{-1cm}
\setlength{\paperheight}{29.7cm}
\setlength{\textheight}{25.3cm}

\begin{document}

% =================================================
\title{Design of the Curve Fitting Library}

\author{Yuwei Liang 3230102923
  \thanks{Electronic address: \texttt{liangyuwei631@gmail.com}}}
\affil{(Mathematics and Applied Mathematics 2302), Zhejiang University}

\date{\today}

\maketitle

\begin{abstract}
The report presents the design and implement of polynomial-piecewise spline(PPSpline), BSpline and CurveFitting.
\end{abstract}

% ============================================

\section{PPSpline}

\subsection{Class Hierarchy}
The \texttt{PPSpline} library uses an object-oriented structure. Below is the inheritance hierarchy:

\begin{itemize}
    \item \textbf{\texttt{PPSpline}} (Abstract Base Class)
    \begin{itemize}
        \item \textbf{\texttt{LinearPPSpline}} (Linear interpolation spline)
        \item \textbf{\texttt{CubicPPSpline}} (Base class for cubic splines)
        \begin{itemize}
            \item \textbf{\texttt{CompleteCubicPPSpline}} (Specifies derivatives at boundaries)
            \item \textbf{\texttt{NaturalCubicPPSpline}} (Zero second derivatives at boundaries)
            \item \textbf{\texttt{PeriodicCubicPPSpline}} (Periodic boundary conditions)
        \end{itemize}
    \end{itemize}
\end{itemize}

\subsection{ Class Overview and Functionality}
\begin{itemize}
    \item \texttt{PPSpline}: Abstract base class for general piecewise polynomial splines.
    \begin{itemize}
        \item \textbf{Purpose:} Provides a framework for spline interpolation.
        \item \textbf{Key Methods:}
            \begin{itemize}
                \item \texttt{evaluate(double x)}: Computes the interpolated value at a given point.
                \item \texttt{computeSpline()}: Pure virtual method for deriving spline coefficients.
                \item \texttt{plot()}: Plots the spline within its defined range.
            \end{itemize}
    \end{itemize}
    \item \texttt{LinearPPSpline}: Implements linear interpolation.
    \begin{itemize}
        \item \textbf{Purpose:} Creates a linear interpolation spline between nodes.
        \item \textbf{Key Methods:}
            \begin{itemize}
                \item \texttt{computeSpline()}: Calculates linear coefficients \texttt{a} and \texttt{b} for each segment.
            \end{itemize}
    \end{itemize}
    \item \texttt{CubicPPSpline}: Base class for cubic splines.
    \begin{itemize}
        \item \textbf{Purpose:} Provides a foundation for cubic spline calculations.
        \item \textbf{Key Variables:}
            \begin{itemize}
                \item \texttt{lambdas, mus, Ks}: Arrays storing intermediate parameters for cubic spline calculation.
            \end{itemize}
        \item \textbf{Key Methods:}
            \begin{itemize}
                \item \texttt{computePara()}: Precomputes parameters like divided differences, lambdas, and mus.
            \end{itemize}
    \end{itemize}
    \item \texttt{CompleteCubicPPSpline}: Specifies boundary derivatives for cubic splines.
    \begin{itemize}
        \item \textbf{Purpose:} Defines a cubic spline with boundary derivatives provided.
        \item \textbf{Key Methods:}
            \begin{itemize}
                \item \texttt{computeSpline()}: Solves a tridiagonal system to determine cubic coefficients.
            \end{itemize}
    \end{itemize}
    \item \texttt{NaturalCubicPPSpline}: Sets second derivatives to zero at boundaries.
    \begin{itemize}
        \item \textbf{Purpose:} Constructs a cubic spline with natural boundary conditions.
        \item \textbf{Key Methods:}
            \begin{itemize}
                \item \texttt{computeSpline()}: Similar to \texttt{CompleteCubicPPSpline} but sets boundary second derivatives to zero.
            \end{itemize}
    \end{itemize}
    \item \texttt{PeriodicCubicPPSpline}: Ensures periodicity in cubic splines.
    \begin{itemize}
        \item \textbf{Purpose:} Creates a spline with periodic boundary conditions.
        \item \textbf{Key Methods:}
            \begin{itemize}
                \item \texttt{computeSpline()}: Solves a cyclic tridiagonal system for periodic cubic coefficients.
            \end{itemize}
    \end{itemize}
\end{itemize}

\subsection{Algorithm Details(cubic spline)}

\subsubsection{Complete Cubic PPSpline}

For a complete cubic spline, we aleady have $m_1 = f'(a)$ and $m_n = f'(b)$, where $a$ and $b$ are the boundaries. Then we can solve the following linear system to get the coefficients $m_i$ for each segment $[x_{i-1}, x_i]$.

\[
\begin{bmatrix}
2 & \mu_2 &  &  &  &  \\
\lambda_3 & 2 & \mu_3 &  &  &  \\
 & \ddots & \ddots & \ddots &  &  \\
 &  & \lambda_i & 2 & \mu_i &  \\
 &  &  & \ddots & \ddots & \ddots \\
 &  &  &  & \lambda_{N-1} & 2 & \mu_{N-2} \\
 &  &  &  &  & \lambda_{N-2} & 2
\end{bmatrix}
\begin{bmatrix}
m_2 \\
m_3 \\
\vdots \\
m_i \\
\vdots \\
m_{N-2} \\
m_{N-1}
\end{bmatrix}
=
\mathbf{b}
\]
where $\lambda_i = \frac{h_{i-1}}{h_{i-1} + h_i}$, $\mu_i = 1 - \lambda_i$, $K_i = \frac{f(x_{i+1}) - f(x_i)}{h_i}$, $h_i = x_{i+1} - x_i$; and the elements of $\mathbf{b} = (b_2, b_3, \dots, b_{N-1})$ are
\[b_2 =  3 \mu_2 K_2 + 3\lambda_2 K_1 -\lambda_2 m_1, \quad b_{N-1} = 3 \mu_{N-1} K_{N-1} + 3\lambda_{N-1} K_{N-2} - \mu_{N-1} m_n \]
\[\forall i = 3, 4 ,\ldots, N-2, \quad 
b_i = 3 \mu_i K_i + 3\lambda_i K_{i-1} \]
\subsubsection{Natural Cubic PPSpline}

For a natural cubic spline, we set $M_1 = M_n = 0$. Then we can solve the following linear system to get the coefficients $M_i$ for each segment $[x_{i-1}, x_i]$.

\[
\begin{bmatrix}
2 & \lambda_2 &  &  &  &  \\
\mu_3 & 2 & \lambda_3 &  &  &  \\
 & \ddots & \ddots & \ddots &  &  \\
 &  & \mu_i & 2 & \lambda_i &  \\
 &  &  & \ddots & \ddots & \ddots \\
 &  &  &  & \mu_{N-2} & 2 & \lambda_{N-2} \\
 &  &  &  &  & \mu_{N-1} & 2
\end{bmatrix}
\begin{bmatrix}
M_2 \\
M_3 \\
\vdots \\
M_i \\
\vdots \\
M_{N-2} \\
M_{N-1}
\end{bmatrix}
=
\mathbf{b}
\]
where the elements of $\mathbf{b} = (b_2, b_3, \dots, b_{N-1})$ are
\[b_i = 6 f[x_{i-1},x_i,x_{i+1}], \quad \forall i = 2,3,\ldots, N-1  \]
\subsubsection{Periodic Cubic PPSpline}
For periodic cubic spline, we can get the following linear system to get the coefficients $m_i$ for each segment $[x_{i-1}, x_i]$.

\[
\begin{bmatrix}
2 & \mu_1 &  &  &  & \lambda_1 \\
\lambda_2 & 2 & \mu_2 &  &  &  \\
 & \ddots & \ddots & \ddots &  &  \\
 &  & \lambda_{i-1} & 2 & \mu_i &  \\
 &  &  & \ddots & \ddots & \lambda_{N-1} \\
\mu_N &  &  &  & \lambda_N & 2
\end{bmatrix}
\begin{bmatrix}
m_1 \\
m_2 \\
\vdots \\
m_i \\
\vdots \\
m_N
\end{bmatrix}
=
\mathbf{b},
\]
the off-diagonal elements of matrix are
\[
\mu_1 = \frac{t_{N+1} - t_N}{t_{N+1} - t_N + t_2 - t_1}, \quad
\lambda_1 = \frac{t_2 - t_1}{t_{N+1} - t_N + t_2 - t_1},
\]
and the elements of $\mathbf{b}$ are $b_i = 3\mu_i K_i + 3\lambda_i K_{i-1}$ 

\section{BSpline}

\subsection{Class Hierarchy}

\begin{itemize}
    \item \textbf{\texttt{Function}} (Base Class for all functions)
    \item \textbf{\texttt{Bbase}}: Represents a single B-spline basis function.
    \item \textbf{\texttt{BSpline}}: Abstract base class for general B-spline interpolation.
    \begin{itemize}
        \item \textbf{\texttt{LinearBSpline}}: Implements linear B-spline interpolation.
        \item \textbf{\texttt{QuarticBSpline}}: Implements quartic B-spline interpolation.
        \item \textbf{\texttt{CubicBSpline}}: Implements cubic B-spline interpolation.
        \begin{itemize}
            \item \textbf{\texttt{CompleteCubicBSpline}}: Allows setting first derivatives at boundaries.
            \item \textbf{\texttt{NaturalCubicBSpline}}: Ensures second derivatives are zero at boundaries.
            \item \textbf{\texttt{PeriodicCubicBSpline}}: Ensures seamless periodic boundary conditions.
        \end{itemize}
    \end{itemize}
\end{itemize}

\subsection{Summary of Functionality}

\begin{itemize}
    \item \textbf{\texttt{Bbase}}:
    \begin{itemize}
        \item Defines B-spline basis functions of arbitrary degree.
        \item Supports evaluation, first derivative, and second derivative computations.
    \end{itemize}
    \item \textbf{\texttt{BSpline}}:
    \begin{itemize}
        \item Abstracts general B-spline interpolation and supports initialization with nodes and values.
        \item Provides methods for visualization and evaluation.
    \end{itemize}
    \item \textbf{Derived Classes:}
    \begin{itemize}
        \item Implement specific spline variations by defining coefficients and boundary conditions.
    \end{itemize}
\end{itemize}

\subsection{Key Class Members and Functions}

\subsubsection{\texttt{Bbase} Class}

\textbf{Key Members:}
\begin{itemize}
    \item \texttt{std::vector<double> nodes\_}: The knots defining the basis function.
    \item \texttt{int degree\_}: Degree of the basis function.
\end{itemize}

\textbf{Key Functions:}
\begin{itemize}
    \item \texttt{double evaluate(double x) const}:
    Computes the value of the basis function at $x$.
    \item \texttt{double derivative(double x) const}:
    Computes the first derivative at $x$.
    \item \texttt{double secondDerivative(double x) const}:
    Computes the second derivative at $x$.
\end{itemize}

\subsubsection{\texttt{BSpline} Class}

\textbf{Key Members:}
\begin{itemize}
    \item \texttt{std::vector<double> nodes\_}: Knots of the B-spline.
    \item \texttt{std::vector<double> coefficients\_}: Coefficients of the B-spline.
    \item \texttt{std::vector<Bbase> basis\_}: Precomputed basis functions.
    \item \texttt{int degree\_}: Degree of the B-spline.
    \item \texttt{int n}: Number of nodes.
\end{itemize}

\textbf{Key Functions:}
\begin{itemize}
    \item \texttt{BSpline(const std::vector<double>\& nodes, const std::vector<double>\& values)}:
    Initializes the B-spline with nodes and values.
    \item \texttt{double evaluate(double x) const}:
    Computes the spline value at $x$ using:
    \[
    S(x) = \sum_{i} c_i B_i(x)
    \]
    \item \texttt{void generate\_extended\_points()}:
    Creates extended nodes for proper basis function definition.
    \item \texttt{void generate\_basis()}:
    Precomputes basis functions for spline evaluation.
\end{itemize}

\subsubsection{Derived Classes}

\textbf{LinearBSpline:}
\begin{itemize}
    \item Sets \texttt{degree\_ = 1}.
    \item \texttt{void computeSpline()}:
    Uses values directly as coefficients.
\end{itemize}

\textbf{QuarticBSpline:}
\begin{itemize}
    \item Sets \texttt{degree\_ = 2}.
    \item \texttt{void computeSpline()}:
    Computes quartic coefficients using a tridiagonal solver.
\end{itemize}

\textbf{CubicBSpline:}
\begin{itemize}
    \item Sets \texttt{degree\_ = 3}.
    \item Variants handle different boundary conditions.
\end{itemize}

\subsection{Core Algorithm Processes}

\subsubsection{B-Spline Basis Function Evaluation}

The B-spline basis functions are computed recursively:

\begin{itemize}
    \item \textbf{Base Case:}
    \[
    B^0_i(x) = 
    \begin{cases} 
    1, & t_i \leq x < t_{i+1} \\
    0, & \text{otherwise}
    \end{cases}
    \]
    \item \textbf{Recursive Formula:}
    \[
    B^k_i(x) = \frac{x - t_i}{t_{i+k} - t_i} B^{k-1}_i(x) + \frac{t_{i+k+1} - x}{t_{i+k+1} - t_{i+1}} B^{k-1}_{i+1}(x)
    \]
\end{itemize}

\subsubsection{Spline Coefficient Computation}
For cubic B-splines, the function `solveUniqueTridiagonal` is used to solve a specific linear system \(Ax=d\) where

\[
A = 
\begin{bmatrix}
s_1 & s_2 & s_3 &  &  &  &  &  \\
a_1 & b_1 & c_1 &  &  &  &  &  \\
 & a_2 & b_2 & c_2 &  &  &  &  \\
 &  & \ddots & \ddots & \ddots &  &  &  \\
 &  &  & a_i & b_i & c_i &  &  \\
 &  &  &  & \ddots & \ddots & \ddots &  \\
 &  &  &  &  & a_{N-1} & b_{N-1} & c_{N-1} \\
 &  &  &  &  &  & a_N & b_N & c_N \\
 &  &  &  &  &  & t_1 & t_2 & t_3
\end{bmatrix}
\]
The function will first transform the matrix into a traditional tridiagonal matrix and then solve it using the Thomas algorithm.

For Complete Cubic BSpline and Natural Cubic BSpline, the matrix we will solve will be in this form:

And for periodic cubic BSpline, the idea is Similar with periodic cubic PPSpline.

\subsection{Visual Summary}

\begin{tcolorbox}[colback=white!95!blue,colframe=blue!40!black,title=Algorithm Overview]
Input: Nodes, Values, Degree \\
$\downarrow$ \\
Extend Nodes \\
$\downarrow$ \\
Generate Basis Functions \\
$\downarrow$ \\
Setup System of Equations \\
$\downarrow$ \\
Solve Coefficients\\
$\downarrow$ \\
Evaluate Spline: $S(x) = \sum c_i \cdot B_i(x)$
\end{tcolorbox}

\section{Curve Fitting}

\subsection{Class Hierarchy}

\begin{itemize}
    \item \textbf{\texttt{Curve}} (Abstract Base Class):
    \begin{itemize}
        \item Provides an interface for parametric curves.
        \item Supports curve evaluation, tangent computation, and visualization.
    \end{itemize}
    \item \textbf{\texttt{CurveFitting}}:
    \begin{itemize}
        \item Implements parametric curve fitting using cubic B-splines.
        \item Automatically determines and fits either \texttt{NaturalCubicBSpline} or \texttt{PeriodicCubicBSpline} for both $x$ and $y$ coordinates.
    \end{itemize}
\end{itemize}

\subsection{Summary of Functionality}

\begin{itemize}
    \item \textbf{\texttt{Curve}}:
    \begin{itemize}
        \item Supports parametric evaluation and tangent computation.
        \item Provides a method to output sampled points for visualization.
    \end{itemize}
    \item \textbf{\texttt{CurveFitting}}:
    \begin{itemize}
        \item Fits a smooth parametric curve to given data or another curve.
        \item Automatically constructs natural or periodic cubic B-splines for interpolation.
    \end{itemize}
\end{itemize}


\subsection{Key Class Members and Functions}

\subsubsection{\texttt{Curve} Class}

\textbf{Key Members:}
\begin{itemize}
    \item \texttt{double tStart, tEnd}: Start and end parameters of the curve.
\end{itemize}

\textbf{Key Functions:}
\begin{itemize}
    \item \texttt{virtual Point operator()(double t) const = 0}:
    Pure virtual function to evaluate the curve at parameter $t$.
    \item \texttt{Point tangent(double t) const}:
    Computes the tangent vector at parameter $t$ using central differences:
    \[
    \text{tangent}(t) = \frac{P(t + h) - P(t - h)}{2h}.
    \]
    \item \texttt{void plot(std::ostream\& os, size\_t samples = 300) const}:
    Outputs sampled points of the curve for visualization.
\end{itemize}

\subsubsection{\texttt{CurveFitting} Class}

\textbf{Key Members:}
\begin{itemize}
    \item \texttt{std::vector<Point> points\_}: Stores points sampled from the curve.
    \item \texttt{std::vector<double> cumulativeLengths\_}: Stores cumulative arc lengths for parameterization.
    \item \texttt{std::unique\_ptr<CubicBSpline> splineX\_}: Spline for the $x$ coordinate.
    \item \texttt{std::unique\_ptr<CubicBSpline> splineY\_}: Spline for the $y$ coordinate.
\end{itemize}

\textbf{Key Functions:}
\begin{itemize}
    \item \texttt{CurveFitting(const Curve\& curve, int numControl)}:
    Fits a parametric curve with a specified number of control points.
    \item \texttt{Point operator()(double t) const override}:
    Evaluates the fitted curve at parameter $t$ using:
    \[
    x = \text{splineX\_}(t), \quad y = \text{splineY\_}(t).
    \]
    \item \texttt{int control\_points\_size() const}:
    Returns the number of control points.
\end{itemize}

% ============================================

\subsection{Core Algorithm Processes}

\subsubsection{Curve Sampling and Parameterization}

To fit a curve:
\begin{enumerate}
    \item Sample points along the curve at equal intervals (or based on speed).
    \item Compute cumulative arc lengths to parameterize the curve.
\end{enumerate}

\subsubsection{Spline Fitting}

\begin{enumerate}
    \item Extract $x$ and $y$ coordinates from the sampled points.
    \item Determine the boundary conditions for splines:
    \begin{itemize}
        \item Use \texttt{PeriodicCubicBSpline} if the first and last coordinates are close.
        \item Use \texttt{NaturalCubicBSpline} otherwise.
    \end{itemize}
    \item Fit separate cubic splines for $x$ and $y$ coordinates.
\end{enumerate}

\subsubsection{Curve Evaluation}

Evaluate the curve at parameter $t$ by computing:
\[
x = \text{splineX\_}(t), \quad y = \text{splineY\_}(t).
\]

But unfortunately, the parameter $t$ changed to the arc length $s$.

\subsection{Sphere Curve Fitting}
Actually, I designed this class a half month later than I finish above report. So I'm too lazy to write a complete report. So, just easy.

The core idea is to transform this problem to the one above. We can first select a north pole and rotate to ensure that the north pole isn't on the curve. Then we can map the points on the sphere to the plane using stereographic projection, perform curve fitting on the plane, and finally map the curve back to the sphere and rotate it back.




\end{document}
